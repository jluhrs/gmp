% Preamble
\documentclass{report}
\usepackage{textcomp}
\author{Rob Norris}
\date{\today}
\title{Gemini OSGi Environment}

\begin{document}
\maketitle
\tableofcontents
% 
% \chapter{Deployment Guide}
% 
% \section{Introduction}
% 
% One of the primary motivations for moving to OSGi is that is makes deployment 
% considerably easier. Rather than packaging a complete application for deployment, 
% we can simply prepare a deployment descriptor that describes the application 
% configuration and desired set of bundles, and the runtime can download components as
% needed. Application deployment then becomes trivial:
% 
% \begin{enumerate}
%   
% 	\item Ensure that a Java $1.5$ runtime environment is available, and ensure
% 	that {\tt launcher.jar} is available. Normally this jarfile would be downloaded
% 	from the primary bundle server (\S\ref{bundle-server}).
% 	\begin{quote}\begin{scriptsize}\begin{verbatim}
% 	wget http://hlservices:9999/runtime/launcher/launcher.jar
% 	\end{verbatim}\end{scriptsize}\end{quote}
% 
% 	\item Install the desired application using the launcher. In this case we are
% 	installing an application for Gemini South.
% 	\begin{quote}\begin{scriptsize}\begin{verbatim}
% 	java -jar launcher.jar -h:hlservices -i:the-application
% 	\end{verbatim}\end{scriptsize}\end{quote}
%   
% 	\item Run the application. A server application with redirected output would be
% 	launcher as
% 	\begin{quote}\begin{scriptsize}\begin{verbatim}
% 	java -jar launcher.jar -a:the-application -n &
% 	\end{verbatim}\end{scriptsize}\end{quote}
% 
% \end{enumerate}
% 
% \section{The Problem}
% 
% The goal of the deployment model is to address a set of issues that historically 
% have been common points of fragility.
% 
% \begin{description}
% 
% \item[Environment Variables.] Environment variables are extremely fragile and 
% are a very common point of failure. They must be configured for each 
% \textlangle{}user, host, shell\textrangle{} tuple, and are one of the most 
% common points of failure when setting things up. They are also very easy to screw
% up later.
% 
% \item[Shared Configuration Files.] Shared configuration files are a good idea 
% in principle because they reduce data duplication, but they also tend to become 
% dumping grounds for application-specific properties and are frightening to 
% clean up because it is not always apparent which applications are using which 
% values, and how.
% 
% \item[Shared Resources.] In our deployment environment it is common for the same
% Unix user to run the same application multiple times from multiple hosts, using the
% same shared disk. This makes it difficult to handle user preferences and disk-based
% application state because any instance can clobber the data of any other instance.
% 
% \item[Shell Scripts.] In the Unix world it has become habit to have a shell script
% for every application. At best these scripts introduce another moving part. At worst
% they introduce an untestable, environment-bound micro-application with its own set
% of failure points.
% 
% \item[Command-line Arguments.] Beause the Java VM has not historically had access to
% very much environment information, in an environment-bound environment Java 
% applications would inevitably require extra command-line arguments or {\tt -D} 
% variables. The only practical way to launch an application in this environment is to
% use a shell script.
% 
% \item[Runtime Libraries.] Historically it has become habit to have a shared 
% dumping ground for shared libraries (jarfiles in the case of Java). This can 
% lead to versioning incompatibilities, but also leads to bloat ue to apathy 
% about the tidiness of the dump ({\it ``Who cares if I add another jarfile, 
% there are already so damn many''}\/) and makes it nearly impossible to delete 
% anything.
% 
% \item[Deployment Packaging.] 
% 
% \item[Updates.] 
% 
% 
% \end{description}
% 



\chapter{Build Environment}
\section{Overview}

The {\tt osgi} build tree contains a {\tt build.xml} file and the following 
top-level subtrees.
\begin{description}
	
	\item[\tt app/] contains a top-level directory for each deployable application. 
	Within each of these directories are the configuration files that define the 
	application. See \S\ref{build-applications} for more details on this.
		
	\item[\tt build/] contains the {\tt common.xml} file that ultimately is 
	included by most of the {\tt build.xml} files in the tree, along with project 
	directories for any custom {\it Ant}\/ tasks that may be needed.
		
	\item[\tt bundle/] contains a tree of OSGi bundle projects, grouped in 
	directories simply to keep them organized. A top-level folder called {\tt 
	external} contains standard and Oscar-specific third-party bundle jarfiles that 
	we are using in OCS (to eliminate dependencies on external bundle 
	repositories). See \S\ref{build-bundles} below for details on the structure of 
	a bundle project.
		
	\item[\tt doc/] contains the \LaTeX{} source for this document, as will 
	probably accumulate more documents in the future.
		
	\item[\tt runtime/] contains source code for Oscar, the Gemini application 
	launcher, and a small number of common runtime libraries that are required for 
	complation of bundle projects.

\end{description}

The build itself is very much like the old OCS build, in that you should be 
able to type {\tt ant} anywhere in the tree to build that subtree. Intermediate 
(non-project) directories simply need to include the following trivial antfile 
in order to keep everything hooked together.

\begin{quote}\begin{scriptsize}\begin{verbatim}
<?xml version="1.0"?>
<project default="all">
    <import file="../build.xml"/>
</project>
\end{verbatim}\end{scriptsize}\end{quote}

Note that project directories include a {\tt name} attribute in the {\tt project}
element of their antfile, and intermediate directories do not. The buile looks for
this property as an indication that there is actually something to build. A 
consequence of this convention is that all buildable subtrees must be terminal; it
is not possible to have a project entirely within another project (although they can
certainly be grouped within the same non-project directory). This is by design. 

\section{Bundles (Buildable Projects)}

\subsection{Overview and Layout}
\label{build-bundles}
Ok, first thing is that you need to use Ant $1.6.5$ \ldots{} older versions won't
work.

Most buildable projects in the tree are OSGi bundles, so this discussion will refer
to them as bundles. But the rules are the same either way.

Within the {\tt osgi/bundle/} directory there is a collection of bundles, organized
variously by intermediate directories that contain the trivial {\tt build.xml} 
described above. Bundles themselves are subtrees containing the following top-level
elements. 

\begin{description}
\item[\tt build.xml] contains bundle-specific build information, including at 
least the bundle's symbolic name. See \S\ref{bundle-ant} below for information 
on how to customize the build for your bundle.
\item[\tt manifest.mf] is the bundle's manifest, which will be included 
verbatim as the bundle's {\tt META-INF/MANIFEST.MF} file. Bundles must specify 
all necessary OSGi headers here.
\item[\tt src/] is the default source directory that will be included in the 
build. See \S\ref{bundle-ant} for information about working with multiple 
source directories.
\item[\tt lib/] is the default location where required runtime jars should be 
placed. Any jarfiles found here will be included in the build classpath, and 
the entire {\tt lib/} tree will be include verbatim in the bundle jarfile.
\end{description}

The build produces the following artifacts, which also live in each bundle's top-level
directory. These elements are all created by the {\tt all} target and are deleted
by {\tt clean}. None of these items should be checked into CVS.

\begin{description}
\item[\tt bin/] is where compiled {\tt .class} files go.
\item[\tt {\it project-name}.jar] is the deployable bundle jarfile, containing by 
default any code compiled from {\tt src/}, everything in {\tt lib/}, and of 
course the manifest specified in {\tt manifest.mf}.
\item[\tt {\it project-name}-src.jar] contains the source code for the bundle, 
including by default everything in {\tt src/} and {\tt lib}, plus {\tt 
build.xml} and {\tt manifest.mf}.
\item[\tt doc/api/] is not being generated yet, but probably will be. It will contain
the bundle's JavaDoc.
\end{description}

\subsection{Ant Targets and {\tt build.xml}}
\label{bundle-ant}
The minimal {\tt build.xml} file for a bundle must define a project name, must 
define the default target to be {\tt all}, and must include the parent 
directory's {\tt build.xml} file. This is all that is required if the default 
build settings are sufficient for your bundle.

\begin{quote}\begin{scriptsize}\begin{verbatim}
<?xml version="1.0"?>
<project name="my-project" default="all">
        <import file="../build.xml"/>
</project>
\end{verbatim}\end{scriptsize}\end{quote}

The following variables are defined implicitly by inclusion of the parent
antfile, and may be used as needed in bundle builds. There are actually a few 
more, but these are the useful ones.

\begin{description}
\item[\tt osgi] is the absolute filesystem location for the root of the build tree.
\item[\tt osgi.bundle] is the absolute filesystem location for the root of the {\tt bundle/}
subtree. It is preferable to use this rather than {\tt \$\{osgi\}/bundle/}
just in case the subtree moves.
\end{description}

The following build targets will be defined implicitly by inclusion of the parent
antfile, which will eventually cause {\tt osgi/build/common.xml} to be included.
Most targets can be customized to some degree, or can be modified easily to be
customizable.

\begin{description}
\item[\tt all] is the default target. It calls {\tt compile}, {\tt jar}, and {\tt jar-src}.

\item[\tt compile] builds the source files in {\tt src/} into {\tt bin/}, 
including the local {\tt lib/} jarfiles and the {\tt \$\{osgi\}/runtime/lib} jarfiles 
in the build classpath. Classfiles will be placed in {\tt bin/}. This target is 
customizable via overriding the following variables:
	
	\begin{description}
	\item[\tt build.classpath.extra] is a path-like structure defining any extra 
	classes you wish to include in the build classpath. You will need to use this 
	if you depend on other bundles at build time.
	\item[\tt build.source.extra] is a path-like structure defining any extra 
	source trees that should be compiled in.	
	\end{description}

\item[\tt jar] builds the bundle jar, which will include the contents of {\tt bin}
and {\tt lib}, and will use {\tt manifest.mf} as the jarfile's manifest. This target is 
customizable via overriding the following variables:
	
	\begin{description}
	\item[\tt jar.extra] is a fileset specifying any extra files that should be
	included in the jarfile (but not in {\tt META-INF/}).
	\item[\tt jar.meta] is a fileset specifying any extra files that should be
	included in the jarfile's {\tt META-INF/} directory.
	\end{description}

\item[\tt jar-src] builds the bundle source jar, which will include the contents
of {\tt src} and {\tt lib},  plus {\tt build.xml} and {\tt manifest.mf}. This target is 
customizable via overriding the following variables:
	
	\begin{description}
	\item[\tt jar-src.extra] is a fileset specifying any extra files that should be
	included in the source jarfile.
	\end{description}

\item[\tt clean] removes the build products, including {\tt bin/} and the two
jarfiles.

\end{description}


\section{Deployable Applications}
\label{build-applications}

\subsection{Overview and Layout}
A deployable application is basically just an Oscar runtime with some set of running bundles and
various other settings like system, security, and logging properties. This is the kind of thing that
is easy to do by hand, but even easier to do with the Launcher and Bundle Server. It is important
to understand that there is nothing magical going on; the deployable application infrastructure
is just a tool to save some typing.

Each deployable application is defined a top-level directory under {\tt app/} containing
a {\tt system.properties} file and optionally several other configuration files. For
example, the Weather Station application layout is as follows:
\begin{quote}\begin{scriptsize}\begin{verbatim}
osgi/app/weather-station/about.txt
                         bundle.properties
                         logging.properties
                         system.properties
\end{verbatim}\end{scriptsize}\end{quote}
\noindent The idea here is to eliminate all environmental dependencies, command-line options, and
other moving parts that are traditionally required when an app is deployed. The launcher
uses the configuration files to set up system, logging, and bundle properties before 
passing control to Oscar, removing the need to set everything up via scripting. And you can just
download the whole setup by naming the application. For example, to install, view help,
 and run the sample EPICS weather station for Cerro Pachon:

\begin{quote}\begin{scriptsize}\begin{verbatim}
wget http://manjar:9999/runtime/launcher/launcher.jar
java -jar launcher.jar -h:manjar -i:weather-station
java -jar launcher.jar -a:weather-station -?
java -jar launcher.jar -a:weather-station
\end{verbatim}\end{scriptsize}\end{quote}

\noindent Because configuration files nay be different based on the deployment environment, the Bundle Server
runs these files through the Velocity template engine as they are retrieved by the launcher. This
may sound heinous at first, but it actually works quite well and allows you to define all deployments
in one place. See \S\ref{bundle-server} for details on the template environment.

\subsection{Config Files}

The following config files are supported; anything else will be ignored. Again, please see 
\S\ref{bundle-server} for details on the template environment, which will allow you to customize
config files on the fly.

\begin{description}
\item[\tt about.txt] is a text file that will be shown to the user if the {\tt -?} option is passed
to Launcher when the application is specified. 
\item[\tt bundle.properties] defines properties that will be made available at runtime via the {\tt BundleContext}.
This is the mechanism you should use for passing configuration parameters to bundles at runtime. 
\item[\tt logging.properties] defines JDK logging properties to be used when control is passed to Oscar.
\item[\tt system.properties] is {\it required}\/ and must define at least the Oscar-specific properties 
that specify the bundles to download and start when the application is first launched.
\end{description}

\noindent Once again, see \S\ref{bundle-server} for some examples of how this works in real life.

\chapter{Reference}

\section{Launcher}
\label{launcher}

\subsection{Overview}

Launcher manages the installation, launching, and deinstallation of OSGi 
applications. It is a very thin wrapper for the Oscar OSGi runtime, and simply 
provides a way to download all the relevant configuration files at runtime 
rather than passing them on the commandline.
    
The following options are available, and may be given in any order:
    
\begin{description}    

\item[\tt -?]  Shows the manpage and then exits, or, if {\tt -a} is specified, shows 
descriptive text about the specified application.

\item[\tt -a:{\it name}] Specifies the target application. If no other 
arguments are given, Laucher simply runs the application.

\item[\tt -t]  Lists the installed applications.
  
\item[\tt -i:{\it name}] Installs the named application from the host specified via 
the -h option. -a may also be specified if you wish to use a different name for 
the installed version. This can be useful if you wish to maintain multiple 
versions of the same application.

\item[\tt -n]  Specifies that the application being launched should run in 
non-interactive mode, in which all output is redirected. This option should be 
used for server applications. Note that this option is not compatible with 
applications using the Shell TUI or any other bundle that reads from standard 
input. Normally a Telnet Service would be used instead if interaction is 
required.
    
\item[\tt -u]  Uninstalls the application specified by {\tt -a}.
    
\item[\tt -h:{\it host}] Specifies the host from which the application 
definition should be downloaded. Defaults to localhost.
        
\item[\tt -p:{\it port}] Specifies the port to connect to for installation. 
Defaults to 9999.
            
\item[\tt -v] Verbose output. This will cause launcher to report on what it is 
doing, and what paths and files it is using. Note that this can also be used 
with -n, which will allow you to see where everything is going before output is 
redirected.
            
\item[\tt -b:{\it key}={\it value}] Persistently sets a bundle property to the 
specified value. This is simply a way to    edit your bundle.properties without 
having to find the file and open it in an editor. This option may be specified 
multiple times.
        
\item[\tt -s:{\it key}={\it value}] Persistently sets a system property to the 
specified value. This is similar to {\tt -b} above. Note that you can also pass 
these values as VM options via {\tt -D}. This option may be specified multiple 
times.
    
\item[\tt -L:{\it startlevel}] Persistently sets the threshold at which bundles 
stop being auto- started. By default all bundles specified in the application's 
definition will be installed as needed and started automatically each time the 
application runs.
            
\end{description}            
            
\subsection{Examples}
            
To install application, you must specify at least {\tt -i} and {\tt -h}.
    
\begin{quote}\begin{scriptsize}\begin{verbatim}
java -jar launcher.jar -h:manjar -i:weather-station
\end{verbatim}\end{scriptsize}\end{quote}
        
If you dont want the local application to be called weather-station, you can 
specify an alias with {\tt -a}:

\begin{quote}\begin{scriptsize}\begin{verbatim}    
java -jar launcher.jar -h:manjar -i:weather-station -a:my-app
\end{verbatim}\end{scriptsize}\end{quote}
        
To launch the application:
    
\begin{quote}\begin{scriptsize}\begin{verbatim}    
java -jar launcher.jar -a:my-app
\end{verbatim}\end{scriptsize}\end{quote}
        
If you want to see what launcher is doing, add {\tt -v}. If you want to redirect 
output to files, add {\tt -n}. This redirection is the last thing Launcher does 
before handing control to Oscar, so you can actually specify both {\tt -n} and 
{\tt -v}.
        
To uninstall the application:
    
\begin{quote}\begin{scriptsize}\begin{verbatim}    
java -jar launcher.jar -a:my-app -u
\end{verbatim}\end{scriptsize}\end{quote}
        
To set a bundle property:
    
\begin{quote}\begin{scriptsize}\begin{verbatim}    
java -jar launcher.jar -a:my-app -b:org.osgi.service.http.port=8082
\end{verbatim}\end{scriptsize}\end{quote}
        
Note that you can also just run the application with -v, and Launcher will 
tell you where the config files are, in case you want to edit them by hand.

\subsection{Notes}
This program is designed specifically for deploying Oscar applications, 
and does not re-implement anything that you can already do in Oscar. In 
particular, incremental updates of bundles should be done using the Shell TUI 
or something equivalent like the Telnet Service.

\subsection{Details}
Launcher attempts to download application definitions from the 
specified host/port via http, using the path {\tt /app/{\it appliciation-name}/}. From 
this path the following files are downloaded and used, if available. Otherwise 
Oscar defaults are used.

\begin{description}
\item[\tt policy] - Java security policy. 
\item[\tt bundle.properties] - BundleContext properties. 
\item[\tt system.properties]  - Additional system 
        properties. 
\item[\tt logging.properties] - Logging properties. Note that a 
        {\tt logdir} parameter will be passed along with the HTTP request, in case 
        this value needs to be inserted on the server.
\end{description}

\noindent Launcher maintains its own bundle cache and logging structure distinct from
the normal Oscar and JDK locations, to help with complex deployments.

\begin{quote}\begin{scriptsize}\begin{verbatim}    
~/.ocs/<hostname>/<application-name>/bundle
                                    /conf
                                    /log
\end{verbatim}\end{scriptsize}\end{quote}

\noindent    Most applications can be expected to place their logs in the log directory,
    but this is not strictly required. The files err.log and out.log will be 
    written to the log directory in all cases when -n is used.

\subsection{Compatibility}
    Running an application from the launcher is basically equivalent to running 
    Oscar with a stack of commandline arguments and some shell redirection.
    
\begin{quote}\begin{scriptsize}\begin{verbatim}    
java -jar launcher.jar -a:<app> -n
\end{verbatim}\end{scriptsize}\end{quote}
        
    is roughly equivalent to:
    
\begin{quote}\begin{scriptsize}\begin{verbatim}    
java -Djava.security.manager
     -Dsecurity.policy=~/.ocs/<host>/<app>/conf/policy
     -Djava.util.logging.config.file=~/.ocs/<host>/<app>/conf/logging.properties
     -Doscar.system.properties=~/.ocs/<host>/<app>/conf/system.properties
     -Doscar.bundle.properties=~/.ocs/<host>/<app>/conf/bundle.properties 
     -Doscar.cache.profiledir=~/.ocs/<host>/<app>/bundle 
     -jar oscar.jar
     >~/.ocs/<host>/<app>/log/out.log 
     2>~/.ocs/<host>/<app>/log/err.log
\end{verbatim}\end{scriptsize}\end{quote}
             
\subsection{Bugs and Limitations}
    You can use spaces in application names, but it's non-obvious. You need
    to put the entire option in quotes:
    
\begin{quote}\begin{scriptsize}\begin{verbatim}    
java -jar launcher.jar "-a:My Application"
\end{verbatim}\end{scriptsize}\end{quote}



\section{Applications}

\subsection{Basic}
Basic is a mostly-empty runtime that is useful for development and experimenting;
it is equivalent to the default Oscar runtime more or less. Use aliases to
install it multiple times.

\begin{quote}\begin{scriptsize}\begin{verbatim}    
manjar:~ rnorris$ launcher -a:test -i:basic  
manjar:~ rnorris$ launcher -a:test   

Welcome to Oscar.
=================

-> ps
START LEVEL 10
   ID   State         Level  Name
[   0] [Active     ] [    0] System Bundle (1.0.5)
[   1] [Active     ] [    1] Shell Service (1.0.2)
[   2] [Active     ] [    1] Shell TUI (1.0.0)
[   3] [Active     ] [    1] Bundle Repository (1.1.2)
[   4] [Active     ] [    2] OSGi Util (1.0.0)
[   5] [Active     ] [    2] OSGi Service (1.0.2)
-> 
\end{verbatim}\end{scriptsize}\end{quote}


\subsection{Bundle Server}
\label{bundle-server}
The Bundle Server provides HTTP-based services to support application
deployment through Launcher and Oscar Bundle Repository (OBR). These services
are provided by default on port $9999$.

\subsubsection{OBR XML Service}
The context path {\tt /bundle} returns dynamically-generated content that
describes every bundle jarfile in the {\tt bundle/} build subtree. Any Oscar
runtime can use this feature to install bundles that are in the Gemini source
tree. You just need to have the Bundle Repository installed (this is standard
unless you turn it off) and use the {\tt obr urls} command to point your Oscar
at the repository.

\begin{quote}\begin{scriptsize}\begin{verbatim}    
-> obr urls
http://oscar-osgi.sf.net/repository.xml
-> obr urls http://localhost:9999/bundle
-> obr list

Bundle Repository (1.1.2)
Email Log Handler (0.1.0)
EPICS Service (0.0.1)
...

-> obr install "EPICS Service"
Installing: EPICS Service
-> 
\end{verbatim}\end{scriptsize}\end{quote}


\subsubsection{Application Deployment Service}

The Bundle Server also provides facilities for downloading application definitions; 
this is the server-side support for Launcher. Config files under {\tt app/} are run
through the Velocity template engine before being returned. For information on
using Velocity (it's really very easy) see

\begin{quote}\begin{scriptsize}\begin{verbatim}
http://jakarta.apache.org/velocity/docs/
\end{verbatim}\end{scriptsize}\end{quote}

\noindent Ok, so in order for Velocity to be useful, there need to be objects bound to
Velocity identifiers. The Bundle Server defines the following useful bindings:

\begin{description}
\item[\tt \$obr.root()] expands to the URL for the OBR XML file where the Bundle
Server is running.
\item[\tt \$obr.urls(["Bundle", ...])] expands a list of bundle names to a list of
URLs for the corresponding bundles. This way you don't have to change URLs if
the bundles move around.
\item[\tt \${\it param}\/] expands to the value of {\it param}\/ on the HTTP request;
so if the URL used when requesting the file includes {\tt ...\&foo=bar} then {\tt \$foo}
will expand to {\tt bar}. The Launcher uses this feature when requesting {\tt logging.properties},
passing the desired log directory as {\tt \$logdir}.
\end{description}


\subsubsection{Bundle Launcher Access}

The Bundle Server provides unmodified access to the {\tt bundle/} and {\tt runtime/} trees;
the former for OBR and the latter so you can download the launcher:

\begin{quote}\begin{scriptsize}\begin{verbatim}
wget http://manjar:9999/runtime/launcher/launcher.jar
\end{verbatim}\end{scriptsize}\end{quote}


\subsubsection{Bootstrapping the Bundle Server}

The Bundle Server is itself a deployable application, but getting it to serve
itself up is a bit of a chicken-and-egg problem. Here is how to set it up.

\begin{enumerate}
  
\item I find that it's easiest to set up an alias for launcher that points to
your  dev tree:
\begin{quote}\begin{scriptsize}\begin{verbatim}
alias launcher='java -jar ~/Projects/osgi/runtime/launcher/launcher.jar'
\end{verbatim}\end{scriptsize}\end{quote}

\item Fetch it from another machine and give it an alias.
\begin{quote}\begin{scriptsize}\begin{verbatim}
launcher -h:manjar -i:bundle-server -a:temp
\end{verbatim}\end{scriptsize}\end{quote}

\item Set the bundle property to point to your local repository (dev) root, and
then start up the app.
\begin{quote}\begin{scriptsize}\begin{verbatim}
launcher -a:temp -b:edu.gemini.util.obr.root=/Users/rnorris/Projects/osgi 
launcher -a:temp
\end{verbatim}\end{scriptsize}\end{quote}
If you type ps -l you will see that everything is running but it's pointed
to the other machine.

\item In another window, install the app again, this time from your own running
server. You also need to set the root property.
\begin{quote}\begin{scriptsize}\begin{verbatim}
launcher -i:bundle-server 
launcher -a:bundle-server -b:edu.gemini.util.obr.root=/Users/rnorris/Projects/osgi
\end{verbatim}\end{scriptsize}\end{quote}
Start up the server so it will download all its bundles. {\it You will get a bind
exception because the port is already bound in the other instance}\/. This is
ok.
\begin{quote}\begin{scriptsize}\begin{verbatim}
launcher -a:bundle-server 
\end{verbatim}\end{scriptsize}\end{quote}

\item Shut down both servers now, by typing "shutdown" at both -> prompts.

\item Uninstall the temp version, and start up the new one.
\begin{quote}\begin{scriptsize}\begin{verbatim}
launcher -a:temp -u 
launcher -a:bundle-server
\end{verbatim}\end{scriptsize}\end{quote}

\item If you do a ps -l you will see that it's pointed at itself now. Note that
this means you can't update the OBR Service itself. If this is going to be a
problem, you can fix this by pointing it at the filesystem location rather than
the HTTP location, then turning auto-start down to level 3.
\begin{quote}\begin{scriptsize}\begin{verbatim}
-> uninstall <number for OBR Service> 
-> install file:/Users/rnorris/Projects/osgi/bundle/util/obr-server/obr-server.jar 
-> start <number for new OBR Service> 
-> shutdown

launcher -a:bundle-server -L:3 
launcher -a:bundle-server
\end{verbatim}\end{scriptsize}\end{quote}
If you do ps -l now, everything should look correct.

\item You should now be able to hit your dev repository from any instance of Oscar
as follows:
\begin{quote}\begin{scriptsize}\begin{verbatim}
-> obr urls http://localhost:9999/bundle 
-> obr list
\end{verbatim}\end{scriptsize}\end{quote}

\end{enumerate}


\subsection{Weather Station}



\section{Bundles}

%%%
%%% BUNDLE: LOG EXTRAS
%%%
\subsection{Jini Driver}

%%%
%%% BUNDLE: LOG EXTRAS
%%%
\subsection{Log Extras}

The Log Extras bundle provides extensions to the standard JDK logger.

\subsubsection{Extensions for \tt logging.properties}

\begin{description}

\item[\it com.xyz.foo.\tt xhandlers] is syntactically equivalent to 
{\tt handlers}, with the following additional properties:
	\begin{itemize}
    \item Handlers are installed when the Log Extras bundle is started, and 
    are uninstalled when the bundle is stopped.
    \item Handlers need not be available on the system classpath (unlike the
    standard {\tt handlers} property).
    \item {\tt xhandlers} supports named handlers (see below).
    \end{itemize}

\end{description}

\subsubsection{Email Handler}

The Email Handler supports emailing of log messages. It will normally be
installed via the {\tt xhandlers} syntax:

\begin{quote}\begin{scriptsize}\begin{verbatim}
xhandlers = edu.gemini.util.logging.EmailHandler
\end{verbatim}\end{scriptsize}\end{quote}

\noindent The following configuration values are available and are 
self-explanatory. You will need to set at least {\tt to}, {\tt from}, and
{\tt host}. Normally you will want to set {\tt level} as well.

\begin{quote}\begin{scriptsize}\begin{verbatim}
edu.gemini.util.logging.EmailHandler.bcc =
edu.gemini.util.logging.EmailHandler.cc = 
edu.gemini.util.logging.EmailHandler.debug = false
edu.gemini.util.logging.EmailHandler.formatter =
edu.gemini.util.logging.EmailHandler.from = bugs@gemini.edu
edu.gemini.util.logging.EmailHandler.host = 172.16.5.12
edu.gemini.util.logging.EmailHandler.level = SEVERE
edu.gemini.util.logging.EmailHandler.to = rnorris@gemini.edu, swalker@gemini.edu
\end{verbatim}\end{scriptsize}\end{quote}


\noindent Note that the Email Handler supports named handler syntax, so you can have
multiple instances if you need them.


\subsubsection{Named Handlers}

JDK Logging does not allow you to have more than one instance of a given
handler, which is sometimes useful. So the Log Extras bundle adds this ability.
This only works for handlers that support naming, which means it only works with 
Gemini custom handlers like the Email Handler. Anyway, you can name a handler 
when you declare it:

\begin{quote}\begin{scriptsize}\begin{verbatim}
my.special.log.xhandlers = edu.gemini.util.logging.EmailHandler("special")
\end{verbatim}\end{scriptsize}\end{quote}

\noindent And can specify config stuff for just that instance, which simply overrides the
config property for the unnamed instance, which will be used by default.


\begin{quote}\begin{scriptsize}\begin{verbatim}
edu.gemini.util.logging.EmailHandler("special").cc = president@whitehouse.gov
\end{verbatim}\end{scriptsize}\end{quote}



%%%
%%% BUNDLE: LOG EXTRAS
%%%
\subsection{OBR Server}



\end{document}